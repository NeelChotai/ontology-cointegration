\documentclass{UoYCSproject}
\addbibresource{bibliography.bib}
\author{Neel Chotai}
\title{Evaluating trade viability through semantic web ontologies}
\date{Version 1.0}
\supervisor{Dimitar Kazakov}
\BEng

\begin{document}
\pagenumbering{roman}
\maketitle
\listoffigures
\listoftables
%\renewcommand*{\lstlistlistingname}{List of Listings}
%\lstlistoflistings

\begin{summary}
<do this at the end>
\end{summary}

\chapter{Introduction}
\label{cha:Introduction}

This report marries together the fields of computer science and economics by seeking to identify whether attributes shared between publicly traded companies have a statistically significant effect on their stock price through pairs trading. Given a pair of companies that share certain attributes, such as shareholders or parent company, we expect the ratio of their prices (i.e. the spread) to remain constant over time. On occasion, there will be a divergence in the spread which may be caused by important news, liquidity or volatility. Over time, we expect that this spread will return to normal and thus we can profit from this by making a pairs trade.

Pairs trading is a form of statistical arbitrage where the trader takes out both a short and long position simultaneously. Highly correlated stocks which cointegrate are the best candidates for pairs trading as the price tends to move together over time. Murray describes cointegration humourously using the example of a drunk and her dog \parencite{drunkdog}. Observing the pattern of a drunk after a night of drinking would look completely random, much like the path of a wandering dog - the dog will meander to wherever its nose takes it. If, however, the dog belongs to the drunk and the drunk calls for the dog, the dog will remain close to the drunk - hence the two wander aimlessly, together. This is exactly the concept of cointegration and what we hope to capitalise on.

Information from SEC reports stored in web ontologies, such as directorship, percentage ownership and company affiliation, may be key to finding highly cointegrated pairs. For example, the case may be that two companies which share directors, officers and parent company have a high degree of correlation, cointegrate well and thus are ideal candidates for pairs trading. Using this information and historical stock data, we can test our hypothesis.

Our hypothesis is that companies with a greater number of relationships within our web ontology, populated with information from SEC reports, will make superior candidates for pairs trading. Our methodology will test to see if companies with a large amount of shared attributes cointegrate well, visually display their relationship and compare how well a trading strategy with this information compares to an index fund.

\chapter{Literature Review}
\label{cha:Literature Review}

This chapter will assess the viability of this project by focusing on three main subject areas. We will be looking at the usage of ontologies, the performance of pairs trading and methods for obtaining pairs to decide whether or not cointegration is the optimal strategy.

\section{Ontologies}

Ontologies are a representation of concepts within a domain, the concepts being categories, properties and relationships between data and entities. Simply put, an ontology shows us characteristics of a subject and how these are related to other characteristics and subjects. In our case, the subjects will be people and companies and characteristics will be things like job title, directorship status, and stock ticker. Ontologies have multiple benefits, as have been noted by Uschold and Gruniger \parencite{ontdef}, such as ease of programming implementation, informative visualisation but the principal among them being that ontologies remove the barrier caused by disparate backgrounds, languages, tools and techniques by providing a single unified interface to a dataset. Ontologies also provide more knowledge about the domain of the data, allowing us to more accurately and easily model relationships.

Ontologies have been used in other fields for much the same reason we believe they will be a better source of information than a simple database. Frank examines how ontologies can be used to improve consistency of data coming from many different sources \parencite{ontgeog}. Instead of consistency constraints and database schemas, ontologies easily integrate data from different sources in to a single system, eliminating erroneous data and allowing our dataset to become more extensible lending us an easy way to new data in the future. Smith et al. have described the use of ontologies in the biomedical field for the purpose of disambiguation \parencite{ontbio}. Smith et al. found that ontologies are more consistent and unambiguous compared to a traditional database in the case of medical vocabulary where two completely unrelated procedures or medicines may share the same name or abbreviation. This is particularly useful for us, for example, if multiple people have the same name; ontologies provide a better way to link together information and an easy interface to visualise these relationships.

Research in to using ontologies in finance is limited, proprietary trading methods lead to little public information being available. However, there have been studies in to representing time-series data in ontologies. 

Knowledge graphs? ...

\section{Pairs trading}

Pairs trading is a market neutral trading strategy that allows traders to profit in virtually any market conditions. The strategy is a form of statistical arbitrage, a class of short-term financial strategies which are highly quantitative in nature and employ mean reversion models. Pairs trading, as the name suggests, consists of buying two positions: one asset is bought long and a similar asset is sold short for a higher price \parencite{pairsdesc}. The assumption being that, by the time the assets are delivered, the prices will be close to equal allowing the trader to profit by the amount the assets initially diverged by - this is known as a convergence trade.

Pairs trades have shown to be profitable by simply choosing two stocks which have moved together in the past \parencite{pairshistory}. The logic behind this strategy is straightforward: if a pair of stocks has historically moved together and one has diverged, we can short the underperforming stock and long the outperforming stock. Assuming that history repeats itself and the stocks converge, the trader can make profit from a relatively low-risk trade.

In a study by Huck and Afawubo \parencite{cointsupport} comparing multiple ways to select pairs, cointegration was found to be a the superior method exhibiting high and robust positive alpha. Further research by Do et al. goes further compares Engle and Granger’s 2-step approach with Johansen’s, finding the former to be influenced by the ordering of the variables and, on occasion, returning spurious estimators \parencite{cointcompared}. Further investigation on this will be undertaken in the design section where we compare the two methodologies.

\section{Cointegeration}

Two time series variables are cointegrated if the expected value of the ratio converges to the mean over time. Granger defines cointegration as follows \parencite{cointdef}:
Take two series, $x_{t}$ and $y_{t}$, each of which is integrated of order 1 (\emph{I}(1)) and has no drift or trend in mean. Generally, any linear combination of these series is also \emph{I}(1). However if there exists some constant A, such that

\[z_t = x_t - Ay_t\]

is \emph{I}(0), $x_t$ and $y_t$ are said to be cointegrated with A being the cointegrating parameter. To test for cointegration, there are a few methodologies. The ones we will be focusing on and deciding between are the Johansen test \parencite{johansen} and the augmented Engle-Granger two-step cointegration test (which utilises the augmented Dickey-Fuller test) \parencite{adf}. Both tests return the hedge ratio and p-values, allowing us to reject the null hypothesis of unit root or no cointegration.

I've chosen to use cointegration to select pairs as opposed to correlation is uncertainty. Correlation indicates that two securities are linked but says nothing about their magnitudes. For example, two stocks may climb together but at vastly different rates. Cointegration removes this uncertainty by identifying pairs that do not drift too far away from each other over the long term. <THIS IS MORE SUITABLE IN METHOD/PROBLEM ANALYSIS>

\chapter{Problem Analysis}
\label{cha:Problem Analysis}

<Huck and Afawubo's research applies here in motivation - methods are mostly proprietary>

\section{Objectives}

<state what we wish to achieve>

\section{Motivation}

<state why we're investigating this area>

\section{Requirements}



\section{Problems}

\section{Evaluation criteria}

\begin{figure}[htb]
\begin{center}
\includegraphics[height=3cm]{"./UoY"}
\end{center}
\caption{A figure containing UoY logo and its caption.}
\end{figure}

Donec felis odio, ultricies maximus sem at, fringilla mollis ipsum. Etiam finibus diam vehicula, egestas ante sed, tincidunt neque. Maecenas porttitor euismod ultrices. Duis imperdiet dictum viverra. Donec in ligula quis enim hendrerit euismod. Aliquam vitae lectus massa. Proin id condimentum leo. Cras mollis, diam at faucibus interdum, dui nulla suscipit metus, nec suscipit leo nisl vel nisl. Ut dapibus dignissim iaculis. Etiam et ante sit amet lectus ultrices lobortis a nec turpis.

\begin{table}[htb]
\caption{ A table with its caption.}
\begin{center}
\begin{tabular}{|p{0.3\textwidth}|p{0.6\textwidth}|}
\hline
column A & column B \\\hline
row 1 &
Lorem ipsum dolor sit amet, consectetur adipiscing elit. Pellentesque quis quam at nisi iaculis aliquet vel et quam. \\\hline
row 2 &
Aliquam erat volutpat. Nam at velit a risus faucibus aliquet. Aenean egestas vehicula mi, quis rhoncus sem facilisis in. Interdum et malesuada fames ac ante ipsum primis in faucibus. Sed lobortis lacus quis mauris rutrum auctor. \\\hline
\end{tabular}
\end{center}
\end{table}

Lorem ipsum dolor sit amet, consectetur adipiscing elit. Pellentesque quis quam at nisi iaculis aliquet vel et quam. Aliquam erat volutpat. Nam at velit a risus faucibus aliquet. Aenean egestas vehicula mi, quis rhoncus sem facilisis in. Interdum et malesuada fames ac ante ipsum primis in faucibus. Sed lobortis lacus quis mauris rutrum auctor. Sed tristique imperdiet ligula, eu bibendum ante malesuada ac. Vestibulum leo ex, viverra ut felis quis, posuere aliquam lorem. Curabitur iaculis tempus mauris, vel dictum erat. Proin sed facilisis dui.

Nunc lectus sapien, malesuada a ullamcorper eu, rutrum in tortor. Nam ac diam tellus. Duis nec faucibus erat. Phasellus euismod, elit eget scelerisque consequat, sapien lectus lobortis justo, sit amet posuere ipsum purus lacinia metus. Mauris bibendum eros nulla. In sit amet posuere lacus, sed pharetra dolor. Etiam vulputate nulla et porttitor sagittis. Curabitur id auctor odio. Ut mollis, ante sed sodales tincidunt, lectus erat mattis turpis, nec suscipit orci augue posuere arcu. Sed volutpat sagittis ipsum, dignissim commodo odio tristique non. Maecenas id auctor leo. Maecenas eu mauris id est porttitor tempus. Curabitur id purus ipsum. Ut tempus faucibus tellus, eu dignissim odio tincidunt id. Sed eu lacus dictum, venenatis tellus eu, egestas diam. Nulla quis libero nisl.

Fusce congue mauris leo, nec sagittis justo convallis quis. Maecenas varius tellus vel nulla facilisis accumsan. Nunc in facilisis nulla. Suspendisse ac molestie ante. Duis vulputate sit amet tellus eget efficitur. Cras ut dolor at dolor molestie dignissim. Proin eu massa dapibus, feugiat quam auctor, porttitor urna. Integer porttitor, libero a egestas lacinia, neque leo luctus elit, id iaculis leo quam porta est. Integer ut porttitor tortor.

Donec felis sem, consectetur eu cursus ac, auctor sollicitudin velit. Sed hendrerit, metus et gravida vulputate, urna lacus facilisis neque, sed accumsan tellus nibh vel metus. Ut tempus felis neque, vitae rutrum ligula tristique sit amet. Morbi nec dui lectus. Ut fringilla orci in faucibus sodales. Integer dolor arcu, pellentesque a blandit non, volutpat vitae sem. Curabitur sit amet nunc tincidunt, mollis purus in, luctus eros.

Donec vitae eros purus. Quisque lacinia risus nisi. Maecenas bibendum augue vel vehicula scelerisque. Nullam suscipit risus porttitor ornare sollicitudin. Vestibulum eget felis at quam cursus congue id id odio. Vestibulum convallis eros ac enim pharetra, sed ornare magna imperdiet. Etiam sagittis rhoncus augue. Pellentesque sit amet mi rhoncus, pretium mauris at, suscipit turpis. Aliquam finibus dolor et velit luctus volutpat. Nunc est velit, ultricies sed elementum non, condimentum in risus.


\chapter{Conclusion}
\label{cha:conclusion}
Lorem ipsum dolor sit amet, consectetur adipiscing elit. Pellentesque quis quam at nisi iaculis aliquet vel et quam. Aliquam erat volutpat. Nam at velit a risus faucibus aliquet. Aenean egestas vehicula mi, quis rhoncus sem facilisis in. Interdum et malesuada fames ac ante ipsum primis in faucibus. Sed lobortis lacus quis mauris rutrum auctor. Sed tristique imperdiet ligula, eu bibendum ante malesuada ac. Vestibulum leo ex, viverra ut felis quis, posuere aliquam lorem. Curabitur iaculis tempus mauris, vel dictum erat. Proin sed facilisis dui.

Nunc lectus sapien, malesuada a ullamcorper eu, rutrum in tortor. Nam ac diam tellus. Duis nec faucibus erat. Phasellus euismod, elit eget scelerisque consequat, sapien lectus lobortis justo, sit amet posuere ipsum purus lacinia metus. Mauris bibendum eros nulla. In sit amet posuere lacus, sed pharetra dolor. Etiam vulputate nulla et porttitor sagittis. Curabitur id auctor odio. Ut mollis, ante sed sodales tincidunt, lectus erat mattis turpis, nec suscipit orci augue posuere arcu. Sed volutpat sagittis ipsum, dignissim commodo odio tristique non. Maecenas id auctor leo. Maecenas eu mauris id est porttitor tempus. Curabitur id purus ipsum. Ut tempus faucibus tellus, eu dignissim odio tincidunt id. Sed eu lacus dictum, venenatis tellus eu, egestas diam. Nulla quis libero nisl.

Fusce congue mauris leo, nec sagittis justo convallis quis. Maecenas varius tellus vel nulla facilisis accumsan. Nunc in facilisis nulla. Suspendisse ac molestie ante. Duis vulputate sit amet tellus eget efficitur. Cras ut dolor at dolor molestie dignissim. Proin eu massa dapibus, feugiat quam auctor, porttitor urna. Integer porttitor, libero a egestas lacinia, neque leo luctus elit, id iaculis leo quam porta est. Integer ut porttitor tortor.

Donec felis sem, consectetur eu cursus ac, auctor sollicitudin velit. Sed hendrerit, metus et gravida vulputate, urna lacus facilisis neque, sed accumsan tellus nibh vel metus. Ut tempus felis neque, vitae rutrum ligula tristique sit amet. Morbi nec dui lectus. Ut fringilla orci in faucibus sodales. Integer dolor arcu, pellentesque a blandit non, volutpat vitae sem. Curabitur sit amet nunc tincidunt, mollis purus in, luctus eros.

\appendix
\chapter{Some apendix}
Fusce congue mauris leo, nec sagittis justo convallis quis. Maecenas varius tellus vel nulla facilisis accumsan. Nunc in facilisis nulla. Suspendisse ac molestie ante. Duis vulputate sit amet tellus eget efficitur. Cras ut dolor at dolor molestie dignissim. Proin eu massa dapibus, feugiat quam auctor, porttitor urna. Integer porttitor, libero a egestas lacinia, neque leo luctus elit, id iaculis leo quam porta est. Integer ut porttitor tortor.

\chapter{Another apendix}
Donec felis sem, consectetur eu cursus ac, auctor sollicitudin velit. Sed hendrerit, metus et gravida vulputate, urna lacus facilisis neque, sed accumsan tellus nibh vel metus. Ut tempus felis neque, vitae rutrum ligula tristique sit amet. Morbi nec dui lectus. Ut fringilla orci in faucibus sodales. Integer dolor arcu, pellentesque a blandit non, volutpat vitae sem. Curabitur sit amet nunc tincidunt, mollis purus in, luctus eros.


\printbibliography

\end{document}